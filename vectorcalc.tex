\begin{center}\section{Vector Calculus}\end{center}

This is calculus when applied to vectors and matricies. Basic
multivariable calculus just requires doing the same math as for one
variable, but repeated. This type of calculus uses the interactions
between dimensions that can arise.

\bigskip

\subsection{Cross Product}
\vspace{-1em}
\begin{eqnarray*}
{\vec a} \times {\vec b}
&=& a_1b_2{\vec e_3} - a_1b_3{\vec e_2} - a_2b_1{\vec e_3}
+a_2b_3{\vec e_1} +a_3b_1{\vec e_2} -a_3b_2{\vec e_1} \\
&=& (a_2b_3 - a_3b_2){\vec e_1} + (a_3b_1 - a_1b_3){\vec e_2} +
(a_1b_2 - a_2b_1){\vec e_3}
\end{eqnarray*}

\subsection{Scalar Triple Product}

\begin{eqnarray*}
{\vec a} \cdot {\vec b} \times {\vec c}
&=& a_1(b_2c_3 - b_3c_2) - a_2(b_1c_3 - b_3c_1) + a_3(b_1c_2 - b_2c_1)
\\
&=& \begin{vmatrix}
a_1 & a_2 & a_3 \\
b_1 & b_2 & b_3 \\
c_1 & c_2 & c_3
\end{vmatrix} \\
&=& {\vec b} \cdot {\vec c} \times {\vec a}
= {\vec c} \cdot {\vec a} \times {\vec b}
\end{eqnarray*}

That is, the scalar triple product is a particular form, whose
expansion happens to be the determinant of the three vectors
involved. Notice that the first line is simply the three pieces of the
cross product, each times the corresponding piece of ${\vec a}$.

The magnitude of the cross product of two vectors is the
area of the parallelogram they define. Similarly, the scalar triple
product is the volume of the parallelepiped the three vectors define.

There is also a {\bf Vector Triple Product},
${\vec a} \times {\vec b} \times {\vec c}$.

\subsection{Fields}

A function of position is a {\em field}, whether scalar or vector. 

\subsection{Line Integrals}

A line integral is the integral of a field along a path. With a simple
function, this could give the length of a line or curve. 

Given a particle in a field which exerts force based on position, a
line integral gives the work needed to make the particle move along a
path.

$$ \text{Work} = - \text{Force} \cdot \text{distance} $$

The negative is to account for the force from the field acting one
way, and the work that has to be input to move the particle being the
other.

\[
W = \displaystyle\sum\limits_{i=1}^N{\vec F}(r_i) \cdot \Delta r_i
\text{, or in limit: }
\int_{t=a}^{t=b}{\vec F}({\vec r}) \cdot d{\vec r}
\]

\subsubsection{Example}

{\bf Path:} $ x = t, y = t, z = 2t^2, 0 \leq t \leq 1 $

\noindent {\bf Force:} $ F = (y, x, z) $

\begin{eqnarray*}
W
&=& \int\limits_C (y, x, z) \cdot d{\vec r},\hspace{5ex} {\vec r} = (t, t, 2t^2)
\\
&=& \int\limits_C (y, x, z) \cdot \frac{d{\vec r}}{dt}dt,\hspace{5ex}
\frac{d{\vec r}}{dt} = (1, 1, 4t) \\
&=& \int_0^1(t, t, 2t^2) \cdot (1, 1, 4t) dt = \int_0^1(2t + 8t^3)dt
= 3
\end{eqnarray*}


- Conservative vector field: integral around closed loop = 0

- Surface Integrals, like Line but More D

- Volume Integrals, see a pattern?

- Gradients, Divergence, Curl: The Good Stuff

- Laplacian (of a twice-differentiable scalar field)

- Suffix Notation (Einstein notation)

- Kronecker Delta ``substitution tensor''

- Alternating/Antisymmetric Tensor

- Expressing cross products with these

- Gradiant, Divergence, and Curl in Suffix Notation

- Also Laplacian

- Some useful identities

- Divergance Theorem

- Polar Coordinate Systems

- Using Scalars in Suffix Notation

- Tensors, Quotient Rule

- Symmetric and Anti-Symmetric Tensors

- Isotropic Tensors

- Ohm's Law

- Inertia Tensor

- Maxwell's Equations

