\section{Differential Equations}

Ordinary differential equation (ODE) only uses x, y(x), y'(x). Trying
to get standard $y = f(x)$ form. Look at start, see what form you can
get into. Note: if you make up an arbitrary DE to try to solve, you'll
probably end up trying to do an impossible integration.

\begin{multicols}{2}

\section{y' = X(x)Y(y)}

$$ y' = X(x)Y(y) = \frac{dy}{dx} $$
$$ \frac{1}{Y(y)}dy = X(x)dx $$
$$ \int\frac{1}{Y(y)}dy = \int X(x)dx $$, then solve for y.

\subsection{Example}

$$ y' = \frac{4y}{x} $$
$$ X(x) = \frac{1}{x}, Y(y) = 4y $$
$$ \int\frac{1}{4y}dy = \int\frac{1}{x}dx  $$
$$ \frac{1}{4}\ln|y| = \ln|x| + C $$
$$ \ln|y| = 4\ln|x| + C $$
$$ y = Cx^4 $$
\columnbreak

\section{y' + p(x)y = q(x)}

Make up a useful u(x), multiply both sides with it and pull a fast one
with the multiplication rule, only divide out at the end.

$$ u(x) = e^{\int p(x)dx}, \frac{du}{dx} = p(x)u(x) $$ (because of the
$e^x$)
$$ u(x)(y' + p(x)y) = u(x)q(x) $$
$$ (u(x)y)' = u(x)q(x) $$ (because $(fg)' = f'g * fg'$)
$$ \int (u(x)y)' dx = \int u(x)q(x)dx $$
$$ y = \frac{1}{u(x)}\int u(x)q(x)dx $$

\subsection{Example}

$$ y' - 2xy = x $$
$$ p(x) = -2x, q(x) = 1, u(x) = e^{\int -2xdx} = Ce^{-x^2} $$
$$ y = Ce^{x^2}\int Cxe^{-x^2}dx $$
use integration by parts
$$ y = Ce^{x^2}(-\frac{1}{2}e^{-x^2} + C)$$
$$ y = Ce^{x^2} - \frac{1}{2} $$
\end{multicols}

\section{Second Order ODEs}

$ y'' + p(x)y' + q(x)y = g(x) $ is easiest. If $g(x) = 0$, it's
``homogeneous'' and is even easier. If $p(x)$ and $q(x)$ are
constants, then $y''$ can also have a constant factor, and the
solution is $ y = C_1e^x + C_2e^{-x} $. If there's no $y$ term, set
$u = y', u' = y''$ to make a 1st-order ODE.

Given constants $a$, $b$, and $c$, and $ a*y'' + b*y'+ c*y = 0 $,
$ y = k_1e^{-\frac{x*\sqrt{b^24ac}+b}{2a}} + k_1e^{\frac{x*\sqrt{b^24ac}+b}{2a}} $

\section{Partial Differential Equations}

Assume PDE's solution will have separated variables, so $ f(x, y) =
X(x)Y(y) $. Feed that assumption through the given equation, to try to
get functions of any variable $x$ together with the corresponding
$dx$. Set the equation equal to a constant. Now you can break the
equation into two differential equations in just one variable each.
